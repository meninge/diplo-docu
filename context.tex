
This report is related with a three month internship in the French Alternative Energies and Atomic Energy Commission (CEA)\nomenclature{CEA}{Commicariat à l'Energie Atomique et au Enérgie Alternative} of Grenoble.
The CEA is one of the most important research centre of France and it is composed of four main areas:
\begin{itemize}
    \item Defence and security,
    \item Nuclear energy (fission and fusion)
    \item Technological research for industry (CEA TECH)
    \item Fundamental research in the physical sciences and life sciences.
\end{itemize}


This assistant engineer internship was done in the CEA TECH, in Grenoble, in the institute named LETI \nomenclature{LETI}{Laboratoire d’électronique et de technologie de l’information} in the laboratory LIALP. \nomenclature{LIALP}{Laboratoire Intégration et Atelier Logiciel pour Puces}\\

The LETI is strongly connected to their 350 partners in the industry creating value and innovation. It specialises in nanotechnologies and their applications, from wireless devices and systems, to biology, healthcare and photonics. Moreover with its 1,700 employees and more than 250 students involved in research activities, LETI has a portfolio of 2,800 patents with about 40\% under licences in various field.\\

In particular, the LIALP (Laboratoire Infrastructure et Atelier Logiciel pour Puces) team is part of the DACLE \nomenclature{DACLE}{Département d’Architectures Conception et Logiciel Embarqués} division (Département d’Architectures Conception et Logiciel Embarqués) and is specialised in software development for specific hardware like many-cores processor or embedded micro-controllers. \\

The LIALP laboratory is a pleasant team of about fifteen permanent researchers and fifteen internship, PhD, PostDoc or CDD. The work atmosphere encourages knowledge sharing. Within the team, there are a lot of project in progress, so discussion are rich and varied which is beneficial for discovery and learning. Some people are working on the optimisation of computation of an application by compilation, others are working on another future potential application like obstacle detection and others intent to improve the power consumption or security.