Pour démontrer que notre réseau de neurone est capable de trouver le nombre écrit à la main 
dans une image, nous allons vous faire une démonstration de ces capacités ainsi qu'un démonstration de ces performances. 
La démonstration se décomposera donc en deux étapes.

\subsection{Démonstration du fonctionnement}

\paragraph{Création de l'image à analyser\\}
	L'image à analyser par notre réseau de neurones doit être créée. Ainsi, nous préparerons 
	un image de $28 \times 28$ pixels dans laquelle nous dessinerons un chiffre de 0 à 9. 
	Cette image de 784 pixels sera alors transformée en un tableau de 784 valeurs. 
	Ce tableau, une fois exporté dans notre logiciel de commande de l'IP, 
	pourra être envoyé comme données d'entrée à traiter par le réseau de neurones.
	
\paragraph{Analyse de l'image et prédiction du chiffre par le réseau de neurone\\}
	Le réseau de neurones va donc être utilisé exactement comme avec les données MNIST\cite{lecun2010mnist}, 
	mise à part que l'image source aura été produite par nos soins. A la sortie du réseau 
	nous trouverons alors un vecteur de taille 10. La coordonnée du vecteur dont la 
	valeur est la plus grande sera la valeur prédite par le réseau de neurone. 
	Par exemple, si en sortie du réseau de neurone nous avons $\{991342, 13124, -104, 54, 567, -345789, 1086, 2, 0, -10\}$ 
	alors le réseau de neurone prédit que le chiffre manuscrit présent sur l'image en entrée est un 0.
	
\paragraph{Affichage du résultat\\}
	Pour finir la démonstration, le programme affichera de façon plus conviviale 
	la valeur que le réseau de neurone a prédite et l'image du chiffre dessinée donnée en entrée. 
	De plus, nous afficherons le taux de réussite sur toutes les images essayées.
	
\subsection{Démonstration des performances}
	Pour la démonstration des performances, nous tenterons de traiter 1000 images 
	et de faire la comparaison en temps et en taux de réussite entre notre IP 
	et le programme de référence purement software. A l'issue de ce test, nous afficherons les 
	différentes grandeurs permettant la comparaison.
