%TODO : à compléter

Le réseau de neurones est un composant qui communique avec le logiciel
via le bus AXI.

\subsection{Registres}

Le composant réseau de neurones est composé de 16 registres de 32 bits,
dont l'utilisation est détaillé dans le
tableau~\ref{fig:user_manual_registers}~page~\pageref{fig:user_manual_registers}.

\begin{longtable}{| p{.10\textwidth} | p{.20\textwidth} | p{.10\textwidth} | p{.50\textwidth} |}
	\hline
	Numéro & Read/Write & Taille & Utilisation (bits)\\ \hline
	0 & Read/Write partiel & 32 bits &
	\begin{itemize}
		\item 15-0: Nombre de données par image
		\item 31-16: Nombre maximum de données par image (Read only)
	\end{itemize}\\ \hline
	1 & Read/Write partiel & 32 bits &
	\begin{itemize}
		\item 15-0: Nombre de neurones dans le premier étage
		\item 31-16: Nombre maximum de neurones dans le premier étage (Read only)
	\end{itemize}\\ \hline
	2 & Read/Write partiel & 32 bits &
	\begin{itemize}
		\item 15-0: Nombre de neurones dans le second étage
		\item 31-16: Nombre maximum de neurones dans le second étage (Read only)
	\end{itemize}\\ \hline
	3 & Read/Write partiel & 32 bits &
	Configuration du mode de l'IP :
	\begin{itemize}
		\item 3-0: \begin{itemize}
				\item \verb+0000+: inactif
				\item \verb+0001+: Calcul du réseau de neurones
				\item \verb+0010+: Chargement des poids pour le premier niveau de neurones
				\item \verb+0100+: Chargement des poids pour l'étage de recodage
				\item \verb+1000+: Chargement des poids pour le second niveau de neurones
				\end{itemize}
		\item 08: Reset
		\item 09: \'{E}tat occupé du maître AXI
	\end{itemize}\\ \hline
	4 & Inutilisé & 32 bits & Inutilisé \\ \hline
	5 & Inutilisé & 32 bits & Inutilisé \\ \hline
	6 & Read/Write & 32 bits &
	\begin{itemize}
		\item 31-00: Nombre de valeurs de sorties du composant à écrire dans la mémoire
	\end{itemize}\\ \hline
	7 & Inutilisé & 32 bits & Inutilisé \\ \hline
	8 & Inutilisé & 32 bits & Inutilisé \\ \hline
	9 & Inutilisé & 32 bits & Inutilisé \\ \hline
	10 & Read/Write & 32 bits &
	\begin{itemize}
		\item 31-00: Adresse de la mémoire où lire les poids ou données
	\end{itemize}\\ \hline
	11 & Read/Write & 32 bits &
	\begin{itemize}
		\item 31-00: Adresse de la mémoire où écrire les données de sortie
	\end{itemize}\\ \hline
	12 & Read/Write & 32 bits &
	\begin{itemize}
		\item 31-00: Nombre de bursts pour la lecture dans la mémoire. La lecture commence quand ce registre est écrit.
	\end{itemize}\\ \hline
	13 & Read/Write & 32 bits &
	\begin{itemize}
		\item 31-00: Nombre de bursts pour l'écriture dans la mémoire. L'écriture commence quand ce registre est écrit.
	\end{itemize}\\ \hline
	14 & Read Only & 32 bits &
	\begin{itemize}
		\item 07-00: Nombre de données dans la FIFO entre le premier niveau de neurones et l'étage de recodage.
		\item 15-08: Nombre de données dans la FIFO entre l'étage de recodage et le second niveau de neurones.
		\item 23-16: Nombre de données dans la FIFO après le second niveau de neurones.
	\end{itemize}\\ \hline
	15 & Read Only & 32 bits &
	\begin{itemize}
		\item 31-16: FIFO Ready/Ack signals, in et out: 12 signals
	\end{itemize}\\ \hline
\caption{Registres du composant FPGA réseau de neurone}
\label{fig:user_manual_registers}
\end{longtable}

\subsection{Implantation sur FPGA avec Vivado}

\subsubsection{Ajout de l'IP dans un projet}
Pour utiliser notre IP, il faut l'ajouter dans un projet {\em Xilinx Vivado}.\\

Pour cela, il faut tout d'abord posséder l'IP dans un dossier. Pour notre projet,
il s'agissait du dossier \verb+ip_repo/+ qui contenait le dossier \verb+my_axi_full_master/+
(qui contenait lui-même notre IP). Vous pouvez tout simplement copier ce dernier dossier dans votre propre
dossier contenant vos IPs.

Ensuite, il faut ouvrir le projet et lancer "{\em Tools -> create and package IP}".
Enfin, sélectionner "{\em package a specific directory}" puis choisir "\verb+<dossier_contenant_les_IPs>/my_axi_full_master+".
Vérifier que le périphérique ajouté est bien un "{\em AXI peripheral}" avant de confirmer.

Après cela, il faut rajouter des blocs de façons à obtenir un {\em bloc design} ressemblant à la figure~\ref{fig:user_man_bloc_design} page~\pageref{fig:user_man_bloc_design}
tout du moins en ce qui concerne le bloc {\em myaxifullmaster\_0} (IP réseau de neurones).

\begin{figure}[h!]
	\includegraphics[width=\textwidth]{user_man_bloc_design}
	\caption{{\em Bloc design} d'un projet utilisant l'IP réseau de neurones}
	\label{fig:user_man_bloc_design}
\end{figure}

Pour modifier les paramètres et générer le bitstream, il faut se référer à la
partie~\ref{plan:user_man_modif_param}
page~\pageref{plan:user_man_modif_param}.


\subsubsection{Modification des paramètres du réseau de neurones}
\label{plan:user_man_modif_param}

Pour modifier les paramètres du réseau de neurones (nombre de neurones de chaque
niveau et taille des images), il faut préciser à {\em Vivado} que les fichiers sources
de l'IP ont changés. Pour cela, il faut:
\begin{enumerate}
	\item modifier dans le fichier \verb+myaxifullmaster_v1_0_S00_AXI.vhd+ (situé dans le sous-dossier \verb+hdl/+ de  \verb+<dossier_contenant_les_IPs>/my_axi_full_master+), les paramètres~:
		\begin{itemize}
			\item \verb+LAYER1_FSIZE+ pour changer la taille des images en entrée,
			\item \verb+LAYER1_NBNEU+ pour changer le nombre de neurones dans le premier niveau,
			\item \verb+LAYER2_NBNEU+ pour changer le nombre de neurones dans le deuxième niveau~;
		\end{itemize}
	\item sélectionner l'IP dans la partie "{\em Project Manager}"
		et avec un clic-droit sélectionner l'option "{\em Re-package IP}".
		Dans la partie "{\em Review and package}", cliquer sur "{\em Re-package IP}" afin de confirmer
		les modifications des fichiers sources de l'IP et retourner sur la fenêtre principale de {\em Vivado}.
\end{enumerate}
~\\
Une fois cela fait, il faut générer le nouveau bitstream en cliquant sur "{\em Generate Bitstream}".
Après génération, il faut faire "{\em File-> export Hardware (cocher include bitstream) -> Ok}".
Enfin pour lancer {\em Xilinx SDK} afin de programmer le FPGA et lancer le logiciel embarqué, il faut faire "{\em File-> Launch SDK}".


%TODO: Partie SW et lancement de l'application
\subsection{Utilisation du réseau de neurones}
Une fois que l'IP du réseau de neurones a été intégrée à {\em Vivado}, que le
bitstream a été généré, exporté et que le SDK est lancé, l'utilisateur
peut utiliser des fonctions simples incluses dans le projet pour lancer le
réseau de neurones.
Il peut soit utiliser les données MNIST pour les poids et frames qui sont
déjà dans le projet (fichiers \texttt{net.c} et \texttt{frame.c}) soit utiliser
ses propres données à rentrer dans le fichier \texttt{dataset.h}. \\
Une fonction permet de lancer le processus de calcul sur l'IP et de récupérer
les résultats, pour chaque frame envoyée, des neurones de sortie. Une fonction
identique permet de faire cette opération seulement en logiciel afin de, plus
tard, comparer performances logicielles et matérielles. D'après le
prototype

\begin{lstlisting}
void nn_hardware(uint8_t *frames, uint32_t *results,
		int16_t *weights_level1,
		int16_t *weights_level2,
		int16_t *constants_recode_level1,
		int16_t *constants_recode_level2);
\end{lstlisting}

on envoie à la fonction l'adresse de la première case des tableaux qui
contiennent les images, les poids, les constantes et du tableau qui contient
les résultats du calcul. La fonction

\begin{lstlisting}
void classify(uint32_t *results, uint32_t *classification);
\end{lstlisting}

classe ces résultats et place dans \texttt{classification} pour chaque image
le neurone séléctionné (pour MNIST c'est le chiffre prédit). \\
Une fois que le code est écrit, l'utilisateur peut le compiler sous le SDK
et lancer tout le programme sur la carte (\texttt{Run}) ou l'exécuter pas à pas
(\texttt{Debug}). Les fonctions utilisées affichent du texte grâce au lien série
avec le composant UART.
