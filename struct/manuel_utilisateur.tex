%TODO : à compléter

Le réseau de neurones est un composant qui communique avec le logiciel
via le bus AXI.

\subsection{Registres}

Le composant réseau de neurones est composé de 16 registres de 32 bits,
dont l'utilisation est détaillé dans le
tableau~\ref{fig:user_manual_registers}~page~\pageref{fig:user_manual_registers}.

\begin{longtable}{| p{.10\textwidth} | p{.20\textwidth} | p{.10\textwidth} | p{.50\textwidth} |}
	\hline
	Numéro & Read/Write & Taille & Utilisation (bits)\\ \hline
	0 & Read/Write partiel & 32 bits &
	\begin{itemize}
		\item 15-0: Nombre de données par image
		\item 31-16: Nombre maximum de données par image (Read only)
	\end{itemize}\\ \hline
	1 & Read/Write partiel & 32 bits &
	\begin{itemize}
		\item 15-0: Nombre de neurones dans le premier étage
		\item 31-16: Nombre maximum de neurones dans le premier étage (Read only)
	\end{itemize}\\ \hline
	2 & Read/Write partiel & 32 bits &
	\begin{itemize}
		\item 15-0: Nombre de neurones dans le second étage
		\item 31-16: Nombre maximum de neurones dans le second étage (Read only)
	\end{itemize}\\ \hline
	3 & Read/Write partiel & 32 bits &
	Configuration du mode de l'IP :
	\begin{itemize}
		\item 3-0: \begin{itemize}
				\item \verb+0000+: inactif
				\item \verb+0001+: Calcul du réseau de neurones
				\item \verb+0010+: Chargement des poids pour le premier niveau de neurones
				\item \verb+0100+: Chargement des poids pour l'étage de recodage
				\item \verb+1000+: Chargement des poids pour le second niveau de neurones
				\end{itemize}
		\item 08: Reset
		\item 09: \'{E}tat occupé du maître AXI
	\end{itemize}\\ \hline
	4 & Inutilisé & 32 bits & Inutilisé \\ \hline
	5 & Inutilisé & 32 bits & Inutilisé \\ \hline
	6 & Read/Write & 32 bits &
	\begin{itemize}
		\item 31-00: Nombre de valeurs de sorties du composant à écrire dans la mémoire
	\end{itemize}\\ \hline
	7 & Inutilisé & 32 bits & Inutilisé \\ \hline
	8 & Inutilisé & 32 bits & Inutilisé \\ \hline
	9 & Inutilisé & 32 bits & Inutilisé \\ \hline
	10 & Read/Write & 32 bits &
	\begin{itemize}
		\item 31-00: Adresse de la mémoire où lire les poids ou données
	\end{itemize}\\ \hline
	11 & Read/Write & 32 bits &
	\begin{itemize}
		\item 31-00: Adresse de la mémoire où écrire les données de sortie
	\end{itemize}\\ \hline
	12 & Read/Write & 32 bits &
	\begin{itemize}
		\item 31-00: Nombre de bursts pour la lecture dans la mémoire. La lecture commence quand ce registre est écrit.
	\end{itemize}\\ \hline
	13 & Read/Write & 32 bits &
	\begin{itemize}
		\item 31-00: Nombre de bursts pour l'écriture dans la mémoire. L'écriture commence quand ce registre est écrit.
	\end{itemize}\\ \hline
	14 & Read Only & 32 bits &
	\begin{itemize}
		\item 07-00: Nombre de données dans la FIFO entre le premier niveau de neurones et l'étage de recodage.
		\item 15-08: Nombre de données dans la FIFO entre l'étage de recodage et le second niveau de neurones.
		\item 23-16: Nombre de données dans la FIFO après le second niveau de neurones.
	\end{itemize}\\ \hline
	15 & Read Only & 32 bits &
	\begin{itemize}
		\item 31-16: FIFO Ready/Ack signals, in et out: 12 signals
	\end{itemize}\\ \hline
\caption{Registres du composant FPGA réseau de neurone}
\label{fig:user_manual_registers}
\end{longtable}
