\nomenclature{FPGA}{Field-Programmable Gate Array : circuits intégrés re-programmables}
\nomenclature{IP}{Intellectual Property : bloc hardware réutilisable ayant une fonctionnalité spécifique}
\nomenclature{MNIST}{Mixed National Institute of Standards and Technology : Base de données de caractères manuscrits servant de tests à de nombreux algorithmes de reconnaissance d'écriture}
\nomenclature{DSP}{Digital Signal Processing : Composant électronique disponible dans le FPGA permettant de faire des multiplications et additions de façon optimisées}
\nomenclature{FIFO}{First In First Out : structure de données permettant de stocker des données et de les restituer dans l'ordre d'arrivée}
\nomenclature{BRAM}{Block Random Access Memory : mémoire spécifiques Xilinx pour les cellules FPGA}
\nomenclature{LUT}{Look Up Table : table de correspondances pour programmer le FPGA}
\nomenclature{DDR}{Double Data Rate : type de mémoire}
\nomenclature{FSM}{Finite State Machine : machine à états finis}
\nomenclature{CIME}{Centre Interuniversitaire de MicroElectronique et nanotechnologies}
\nomenclature{CPU}{Central Processing Unit}
\nomenclature{GPU}{Graphics Processing Unit}
\nomenclature{ASIC}{Application Specific Integrated Circuit}
\nomenclature{TIMA}{Techniques de l'Informatique et de la Microélectronique pour l'Architecture des systèmes intégrés}
\nomenclature{AXI}{Advanced eXtensible Interface : bus de donnée}
\nomenclature{SDK}{Software Development Kit : outil de développement logiciel}
\nomenclature{UART}{Universal Asynchronous Receiver Transmitter : composant permettant de gérer la communication par liaision série entre la carte et l'ordinateur}
\nomenclature{HDL}{Hardware Description Language : description du comportement matériel attendu par un langage de description tel que VHDL ou Verilog}
\nomenclature{PC}{Personnal Computer : ordinateur personnel}
\nomenclature{DMA}{Direct Memory Access : composant permettant des transferts de mémoire direct sans passer
par le CPU} 
