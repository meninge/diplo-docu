\` {A} la fin de ce projet, nous disposons d'une IP réseau de neurones très performante (environ 450 fois plus rapide
qu'un logiciel implémentant le même réseau de neurones). Celle-ci est générique~: il suffit de modifier quelques constantes
dans un fichier VHDL et dans les fichiers sources du logiciel pour changer les paramètres du réseau de neurones (nombre de neurones
de chaque niveau, taille des images d'entrée et nombre d'images). De même, il est configurable dynamiquement, sans avoir à
générer à nouveau le bitstream~:
il est possible via des appels de fonctions de changer la configuration de chaque niveau de neurones et de l'étage de recodage.

Cependant, il est possible d'obtenir des performances encore meilleures (un facteur d'amélioration environ égal à 2) en optimisant la machine à
états de contrôle de neurones. Malheureusement, nous n'avons pas eu le temps d'effectuer cette dernière et de la valider en simulation et sur carte
dans le temps imparti. En effet, nous avons rencontré des problèmes innatendus qui nous ont fortement freiné dans notre progression.
Néanmoins, nous avons réussi à remplir les objectifs du projet à temps en proposant une IP réseau de neurones générique, correcte,
configurable dynamiquement et beaucoup plus performante que le logiciel.\\
~\\
L'IP développée durant ce projet pourra permettre au laboratoire TIMA de comparer les performances
de leur réseau de neurones ternaires à un réseau de neurones plus classqiue sur un même matériel.
De même, il serait intéressant de comparer les performances de notre réseau à des alternatives
telles que {\em SpiNNaker\cite{painkras2013spinnaker}} de l'université de Manchester, ou bien {\em TrueNorth} d'IBM.
