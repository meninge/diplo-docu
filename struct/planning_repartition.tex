Pour travailler de façon efficace, il est essentiel
d'établir les dépendances entre les tâches et de les répartir entre les membres
du groupe.

\subsection{Planning}

Sur la figure~\ref{fig:PlanningPrevision}~page~\pageref{fig:PlanningPrevision},
les tâches ont été réparties temporellement sur la période du projet. Ce planning
prévisionnel a été dans l'ensemble bien respecté durant tout le projet.

\begin{figure}[ht!]
	\begin{ganttchart}
		[y unit title=0.4cm,
			y unit chart=0.5cm,
			vgrid,hgrid,
			title label anchor/.style={below=-1.6ex},
			title left shift=.05,
			title right shift=-.05,
			title height=1,
			bar/.style={fill=gray!50},
			incomplete/.style={fill=white},
			progress label text={},
			bar height=0.7,
			group right shift=0,
			group top shift=.6,
			group height=.3,
		]
		{1}{20}
		%labels
		\gantttitle{Projet}{20} \\
		\gantttitle{Oct}{4}
		\gantttitle{Nov}{4}
		\gantttitle{Dec}{4}
		\gantttitle{Jan}{4}
		\gantttitle{Feb}{4}
		\ganttnewline

		%tasks
		%0
		\ganttbar{Spécifications}{2}{4} \\
		%1
		\ganttbar{Programme de référence}{3}{4} \\
		%2
		\ganttbar{Implantation de l'IP}{5}{13} \\
		%3
		\ganttbar{Validation en simulation}{6}{14} \\
		%4
		\ganttbar{Validation sur carte}{11}{15} \\
		%5
		\ganttbar{Logiciel de commande}{12}{14} \\
		%6
		\ganttbar{Evaluation des performances}{16}{17} \\
		%7
		\ganttbar{Documentation}{14}{17} \\
		%8
		\ganttbar{Soutenance}{18}{18} \\

		%relations

		\ganttlink{elem0}{elem2}
		\ganttlink{elem0}{elem3}
		\ganttlink{elem4}{elem6}
		\ganttlink{elem5}{elem6}
	\end{ganttchart}
	\caption{Planning prévisionnel}
\label{fig:PlanningPrevision}
\end{figure}

Voici la description des différentes tâches:
\begin{itemize}
	\item \textbf{Spécifications}: écrire les spécifications du
		réseau de neurones sur FPGA, définir son architecture.
	\item \textbf{Programme de référence}: écriture du programme en C
		permettant de comparer les résultats de notre IP à une version
		logicielle. Il permettra aussi lors de l'évaluation de
		performances pour mesurer le facteur d'accélération de notre IP.
	\item \textbf{Implantation de l'IP}: écriture en VHDL des différents
		composants de l'IP.
	\item \textbf{Validation en simulation}: Test benches pour vérifier le
		comportement de nos composants, leurs interactions et le
		fonctionnement global de l'IP.
	\item \textbf{Validation sur carte}: Tests sur carte permettant de
		vérifier le comportement de notre IP sur carte Zybo,
		puis Zedboard.
	\item \textbf{Logiciel de commande}: écriture du logiciel permettant de
		contrôler l'IP.
	\item \textbf{Evaluation des performances}: comparaison de performances
		de notre IP par rapport au logiciel de référence fonctionnant
		sur la même carte.
	\item \textbf{Documentation}: écriture de la documentation comprenant
		le cahier des charges, le planning, le manuel utilisateur,
		les étapes de conception, la validation et la présentation de la
		démonstration.
	\item \textbf{Soutenance}: Soutenance finale incluant une démonstration
		du travail effectué.
\end{itemize}

\subsection{Répartition entre les membres du groupe}

La plupart des tâches étant parallélisables, nous avons donc pu les répartir
entre les membres du groupe.

Voici les occupations principales de chaque personne:
\begin{itemize}
	\item Hugues : \'{E}criture du programme de référence et du logiciel de
		commande, implantation de l'IP, documentation
	\item Lucas : Spécifications, implantation de l'IP, validation sur
		carte, documentation
	\item Paul : Spécifications, implantation de l'IP, validation en
		simulation, documentation
\end{itemize}
~\\
Cependant, certains problèmes majeurs dans l'implantation de l'IP et les essais
sur carte sont devenus bloquants et ont fortement réduit la parallélisation des
tâches en obligeant tous les membres du groupe à chercher une solution aux
problèmes rencontrés.

