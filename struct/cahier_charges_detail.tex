%section{cahier des charges détaillé}

\subsection{Présentation générale du problème}

\subsubsection{Projet}
Le but du sujet est de réaliser un réseau de neurones sur FPGA.

TODO : INSERER SCHEMA CLASSIQUE DE RESEAU DE NEURONES ET UNE BREVE DESCRIPTION

\paragraph{Finalités\\}

Ce réseau de neurones sur FPGA a pour but d'établir une référence pour pouvoir
comparer un réseau de neurones ternaire en cours de développement au laboratoire
TIMA à un réseau de neurones plus classique, tous deux réalisés sur le même 
matériel.

Un objectif secondaire serait de comparer le composant réalisé à d'autres
produits similaires tels que la puce neuromorphique Spinnaker de l'université
de Manchester, ou bien la puce TrueNorth d'IBM.

\subsubsection{Contexte}

Le projet sera réalisé au CIME Nanotech, à Grenoble. 

Quatre heures par semaine sont allouées à la réalisation de ce projet. Cependant,
il est nécessaire de travailler en dehors des horaires scolaires pour terminer 
le projet.

Les tests sur carte FPGA seront réalisés dans un premier temps sur une carte Zybo,
puis une fois la fonctionnalité du composant validée, les tests se poursuivront 
sur carte ZedBoard.

\subsubsection{Énoncé du besoin}
%(finalités du produit pour le futur utilisateur tel que prévu par le demandeur)
Les finalités du produit pour le futur utilisateur tel que prévu par le
demandeur sont:
\begin{itemize}
	\item Le composant doit pouvoir être programmable dynamiquement, 
		c'est-à-dire qu'il doit être possible de changer le nombre de
		neurones par étage et les coefficients de chaque étage
		(niveau de neurones et étage de recode).
	\item Le composant doit être générique, c'est-à-dire qu'il faut pouvoir
		changer simplement le nombre de neurones, des données d'entrées
		et des autres paramètres en changeant une variable dans les
		fichiers HDL du composant.
	\item Le composant doit produire le résultat attendu, c'est-à-dire qu'il
		doit calculer le résultat du réseau de neurones qu'il représente.
\end{itemize}

\subsection{Expression fonctionnelle du besoin}

\subsubsection{Fonctions de service et de contrainte}

\paragraph{Fonctions de service principales\\}
% (qui sont la raison d’être du produit)

Le produit doit calculer le résultat d'une donnée d'entrée soumise à un réseau
de neurones, paramétré selon les coefficients précédemment donnés au composant.

\paragraph{Fonctions de service complémentaires\\}
% (qui améliorent, facilitent ou complètent le service rendu)

\paragraph{Contraintes\\}
% (limitations à la liberté du concepteur-réalisateur)

L'architecture globale du composant et ses interfaces (bus) sont déterminées
par le laboratoire TIMA.

Le composant devra utiliser les cellules DSP du FPGA pour réaliser l'accumulation
d'un neurone.

\paragraph{Décomposition en modules, sous-ensembles\\}

% TODO insérer schéma global (haut-niveau) du composant
TODO : INSERER SCHEMA HAUT NIVEAU DU COMPOSANT
ET UNE EXPLICATION SOMMAIRE

\subsubsection{Critères d’appréciation}
% (en soulignant ceux qui sont déterminants pour l’évaluation des réponses)

Les critères permettant de mesurer la qualité du composant produit sont:
\begin{itemize}
	\item Correction du composant : le résultat doit être celui attendu.
	\item Taux d'utilisation des cellules du FPGA
	\item Performances du composant (fréquence, nombre de cycles pour
		calculer le résultat ...)
	\item Généricité du composant
\end{itemize}

\subsubsection{Niveaux des critères d’appréciation}
% et ce qui les caractérise

\paragraph{Niveaux dont l’obtention est imposée\\}

Il est nécessaire que le composant satisfasse les niveaux de critères suivants:
\begin{itemize}
	\item Correction : le résultat doit être correct
	\item Taux d'utilisation des cellules du FPGA : un neurone doit utiliser
		une cellule DSP.
\end{itemize}

\paragraph{Niveaux souhaités mais révisables\\}

Il est souhaitable que le composant satisfasse les niveaux de critères suivants:
\begin{itemize}
	\item Performances : Le résultat d'un calcul du composant doit être plus
		rapide que son équivalent réalisé sur un processeur classique.
	\item Généricité : Les fichiers HDL du composant doivent pouvoir être
		modifié de façon mineure pour changer les paramètres du réseau
		de neurones.
\end{itemize}


