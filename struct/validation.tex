%TODO : améliorer la description des test benchs de simulation.
%TODO : écrire la partie sur la validation sur carte (préciser l'utilisation des
% registres etc.

La validation est une partie importante de notre projet. En effet, elle permet
d'assurer le bon fonctionnement de notre IP.

Pour valider notre IP, nous avons du avoir recours à deux modes de validation :
la validation en simulation et la validation sur carte.
En effet, le simulateur ne permet pas de simuler notre IP complète avec le bus
AXI et le processeur ARM, il faut donc avoir recours à une validation sur carte
Zybo et Zedboard.

\subsection{Validation en simulation}

La validation en simulation permet de vérifier le comportement de chacun des
composants de notre IP, individuellement et en association avec d'autres
composants.

Les simulations ont été réalisé à l'aide du simulateur de Xilinx Vivado.

\subsubsection{Le neurone}

%TODO: LUCAS
%TODO: à compléter
TODO: LUCAS

\subsubsection{La machine à états}

La machine à états contrôle l'ensemble des neurones d'un niveau de neurones.
Ce test bench permet de vérifier le comportement précis de la machine
à états du niveau de neurones.\\

Durant ce test bench, nous avons vérifié que la machine à états:
\begin{itemize}
	\item est dans l'état attendu tout au long du test bench.
	\item contrôle correctement un neurone, à partir du fonctionnement
	détaillé dans les spécifications.
	\item passe en mode chargement des poids ou accumulation et a le
	comportement attendu dans chacun des cas.
	\item effectue un décalage de la chaîne de miroir à la fin d'une
	accumulation.
\end{itemize}
%TODO: à compléter

\subsubsection{Le niveau de neurones}

Le niveau de neurones est composé d'une machine à états et de neurones.
Ce test bench permet de s'assurer que la machine à états et les neurones
interagissent correctement ensemble et que l'on tient bien compte des buffers
de distribution des signaux.\\

Durant ce test bench, nous avons vérifié que la machine à états:
\begin{itemize}
	\item contrôle correctement plusieurs neurones, en tenant
	compte des buffers de distribution des signaux qui ajoutent un
	délai dans la communication.
	\item passe en mode chargement des poids ou accumulation et a le
	comportement attendu dans chacun des cas.
	\item effectue un décalage de la chaîne de miroir à la fin d'une
	accumulation.
\end{itemize}

Nous avons vérifié que les neurones:
\begin{itemize}
	\item chargent correctement les poids en mode chargement des poids.
	\item en mode accumulation, accumulent correctement les données et
	décalent le résultat dans la chaîne de miroir à la fin de
	l'accumulation.
	\item se passent correctement les données dans leurs miroirs lors
	de la phase de décalage des miroirs, en mode accumulation.
\end{itemize}
%TODO: à compléter

\subsubsection{Le recodage}

L'étage de recodage permet d'ajouter une constante aux valeurs en sortie du
niveau de neurones et de supprimer les valeurs négatives.\\

Durant ce test bench, nous avons vérifié que l'étage de recodage:
\begin{itemize}
	\item charge ses poids en mode chargement des poids.
	\item en mode accumulation, ajoute une constante propre à chaque neurone
	à la donnée d'entrée, vérifie que le résultat est positif, le transmet à
	la FIFO suivante dans ce cas, ou transmet 0 dans le cas contraire.
\end{itemize}
%TODO: à compléter

\subsubsection{Interaction FIFO-niveau de neurones}

Ce test bench permet de s'assurer que la FIFO d'entrée d'un niveau de neurones
est bien contrôlée par le niveau de neurones correspondant.\\

Durant ce test bench, nous avons vérifié que le niveau de neurones:
\begin{itemize}
	\item contrôle correctement la FIFO d'entrée, c'est-à-dire qu'il prend
	les données quand elles sont disponibles, et ce pour les deux modes
	(accumulation et chargement des poids).
	\item contrôle correctement la FIFO de sortie, c'est-à-dire qu'il envoie
	des données à l'intérieur uniquement s'il y a suffisamment de place dans
	le mode accumulation. Dans le mode chargement des poids, il ne doit pas
	y avoir d'envoi de données dans cette FIFO.
\end{itemize}
%TODO: à compléter

\subsubsection{Interaction FIFO-recodage-FIFO}

Ce test bench permet de vérifier qu'un étage de recodage contrôle bien ses FIFOs
d'entrée et de sortie.\\

Durant ce test bench, nous avons vérifié que l'étage de recodage:
\begin{itemize}
	\item contrôle correctement la FIFO d'entrée, c'est-à-dire qu'il prend
	les données quand elles sont disponibles, et ce pour les deux modes
	(accumulation et chargement des poids). Dans le mode accumulation, il
	ne doit prendre des données que s'il y a de la place dans la FIFO de
	sortie.
	\item contrôle correctement la FIFO de sortie, c'est-à-dire qu'il envoie
	des données à l'intérieur uniquement s'il y a suffisamment de place dans
	le mode accumulation. Dans le mode chargement des poids, il ne doit pas
	y avoir d'envoi de données dans cette FIFO.
\end{itemize}
%TODO: à compléter

\subsubsection{Interaction FIFO-niveau de neurones-FIFO-recodage}

Ce test bench vérifie que le pipeline FIFO-niveau de neurones-FIFO-recodage
fonctionne correctement.\\

Durant ce test bench, nous avons vérifié que l'ensemble des composants:
\begin{itemize}
	\item charge ses poids en mode chargement des poids, chacun son tour.
	De plus, ils ne doivent pas se comporter de façon incorrecte quand
	ils ne sont ni en train de charger les poids ni en train d'accumuler.
	\item effectue l'opération d'accumulation en mode accumulation, prend
	les données dans sa FIFO d'entrée et transmet les données en sortie
	uniquement s'il y a de la place.
\end{itemize}
%TODO: à compléter

\subsubsection{Conclusion}

Avec l'ensemble des tests en simulation, nous avons couvert tout le pipeline de
notre IP. En effet, chaque composant a été testé individuellement et nous avons
testé les interactions entre FIFO-niveau de neurones-FIFO, FIFO-recode-FIFO.
Nous avons donc vérifié que le pipeline complet
FIFO-niveau de neurones-FIFO-recodage-FIFO-niveau de neurones-FIFO fonctionne
correctement.

Ces tests nous permettent de couvrir l'ensemble de interactions entre composants
de notre IP.

\subsection{Validation sur carte}
%TODO: à compléter
