La validation est une partie importante de notre projet. En effet, elle permet
d'assurer le bon fonctionnement de notre IP.

Pour valider notre IP, nous avons du avoir recours à deux modes de validation :
la validation en simulation et la validation sur carte.
En effet, le simulateur ne permet pas de simuler notre IP complète avec le bus
AXI et le processeur ARM, il faut donc avoir recours à une validation sur carte
Zybo et Zedboard.

\subsection{Validation en simulation}

La validation en simulation permet de vérifier le comportement de chacun des
composants de notre IP, individuellement et en association avec d'autres
composants.

Les simulations ont été réalisé à l'aide du simulateur de Xilinx Vivado.

\subsubsection{Le neurone}
Pour tester le neurone, il faut que le bench de test simule l'environnement dans lequel
le neurone doit évoluer. C'est-à-dire, les interaction avec les autres neurones, 
la machine à états et les FIFO de données en entrée et en sortie. 
Ainsi dans le bench de test du neurone, nous avons vérifier que le neurone :
\begin{itemize}
	\item puisse écrire dans sa BRAM lorsqu'il est en mode chargement des poids.
	\item puisse lire les coefficients en BRAM quand il est en mode accumulation.
	\item réagisse bien à la demande du passage de token.
	\item fasse bien les bons calculs.
	\item copie bien leur registre d'accumulation dans le registre miroir.
	\item évacue bien la donnée du miroir par la chaine de miroir. 
	\item interagisse bien avec les signaux de contrôle des FIFO en lecture et en écriture. 
\end{itemize} 

\subsubsection{La machine à états}

La machine à états contrôle l'ensemble des neurones d'un niveau de neurones.
Ce test bench permet de vérifier le comportement précis de la machine
à états du niveau de neurones.\\

Durant ce test bench, nous avons vérifié que la machine à états:
\begin{itemize}
	\item est dans l'état attendu tout au long du test bench.
	\item contrôle correctement un neurone, à partir du fonctionnement
	détaillé dans les spécifications.
	\item passe en mode chargement des poids ou accumulation et a le
	comportement attendu dans chacun des cas.
	\item effectue un décalage de la chaîne de miroir à la fin d'une
	accumulation.
\end{itemize}

\subsubsection{Le niveau de neurones}

Le niveau de neurones est composé d'une machine à états et de neurones.
Ce test bench permet de s'assurer que la machine à états et les neurones
interagissent correctement ensemble et que l'on tient bien compte des buffers
de distribution des signaux.\\

Durant ce test bench, nous avons vérifié que la machine à états:
\begin{itemize}
	\item contrôle correctement plusieurs neurones, en tenant
	compte des buffers de distribution des signaux qui ajoutent un
	délai dans la communication.
	\item passe en mode chargement des poids ou accumulation et a le
	comportement attendu dans chacun des cas.
	\item effectue un décalage de la chaîne de miroir à la fin d'une
	accumulation.
\end{itemize}

Nous avons vérifié que les neurones:
\begin{itemize}
	\item chargent correctement les poids en mode chargement des poids.
	\item en mode accumulation, accumulent correctement les données et
	décalent le résultat dans la chaîne de miroir à la fin de
	l'accumulation.
	\item se passent correctement les données dans leurs miroirs lors
	de la phase de décalage des miroirs, en mode accumulation.
\end{itemize}

\subsubsection{Le recodage}

L'étage de recodage permet d'ajouter une constante aux valeurs en sortie du
niveau de neurones et de supprimer les valeurs négatives.\\

Durant ce test bench, nous avons vérifié que l'étage de recodage:
\begin{itemize}
	\item charge ses poids en mode chargement des poids.
	\item en mode accumulation, ajoute une constante propre à chaque neurone
	à la donnée d'entrée, vérifie que le résultat est positif, le transmet à
	la FIFO suivante dans ce cas, ou transmet 0 dans le cas contraire.
\end{itemize}

\subsubsection{Interaction FIFO-niveau de neurones}

Ce test bench permet de s'assurer que la FIFO d'entrée d'un niveau de neurones
est bien contrôlée par le niveau de neurones correspondant.\\

Durant ce test bench, nous avons vérifié que le niveau de neurones:
\begin{itemize}
	\item contrôle correctement la FIFO d'entrée, c'est-à-dire qu'il prend
	les données quand elles sont disponibles, et ce pour les deux modes
	(accumulation et chargement des poids).
	\item contrôle correctement la FIFO de sortie, c'est-à-dire qu'il envoie
	des données à l'intérieur uniquement s'il y a suffisamment de place dans
	le mode accumulation. Dans le mode chargement des poids, il ne doit pas
	y avoir d'envoi de données dans cette FIFO.
\end{itemize}

\subsubsection{Interaction FIFO-recodage-FIFO}

Ce test bench permet de vérifier qu'un étage de recodage contrôle bien ses FIFOs
d'entrée et de sortie.\\

Durant ce test bench, nous avons vérifié que l'étage de recodage:
\begin{itemize}
	\item contrôle correctement la FIFO d'entrée, c'est-à-dire qu'il prend
	les données quand elles sont disponibles, et ce pour les deux modes
	(accumulation et chargement des poids). Dans le mode accumulation, il
	ne doit prendre des données que s'il y a de la place dans la FIFO de
	sortie.
	\item contrôle correctement la FIFO de sortie, c'est-à-dire qu'il envoie
	des données à l'intérieur uniquement s'il y a suffisamment de place dans
	le mode accumulation. Dans le mode chargement des poids, il ne doit pas
	y avoir d'envoi de données dans cette FIFO.
\end{itemize}

\subsubsection{Interaction FIFO-niveau de neurones-FIFO-recodage}

Ce test bench vérifie que le pipeline FIFO-niveau de neurones-FIFO-recodage
fonctionne correctement.\\

Durant ce test bench, nous avons vérifié que l'ensemble des composants:
\begin{itemize}
	\item charge ses poids en mode chargement des poids, chacun son tour.
	De plus, ils ne doivent pas se comporter de façon incorrecte quand
	ils ne sont ni en train de charger les poids ni en train d'accumuler.
	\item effectue l'opération d'accumulation en mode accumulation, prend
	les données dans sa FIFO d'entrée et transmet les données en sortie
	uniquement s'il y a de la place.
\end{itemize}

\subsubsection{Conclusion}

Avec l'ensemble des tests en simulation, nous avons couvert tout le pipeline de
notre IP. En effet, chaque composant a été testé individuellement et nous avons
testé les interactions entre FIFO-niveau de neurones-FIFO, FIFO-recode-FIFO.
Nous avons donc vérifié que le pipeline complet
FIFO-niveau de neurones-FIFO-recodage-FIFO-niveau de neurones-FIFO fonctionne
correctement.

Ces tests nous permettent de couvrir l'ensemble de interactions entre composants
de notre IP.

\subsection{Validation sur carte}
%TODO: à compléter
%TODO: LUCAS si tu veux rajouter des trucs ou affiner

Suite aux tests réalisés en simulation et à l'impossibilité de simuler
l'ensemble de la carte, nous avons dû réalisé des tests sur cartes.

Ces tests nous permettent de tester le fonctionnement conjoint du logiciel, du
bus AXI et de notre IP. Cependant, ces tests sont long (il faut générer le
bitstream à charger dans la carte à chaque fois que l'on veut modifier le test ou
en effectuer un autre). De plus, il est impossible d'observer l'ensemble du
fonctionnement de notre IP une fois chargée dans le FPGA. Nous avons du avoir
recours à des registres non utilisés pour récupérer des données et contrôler
plus finement notre IP. Le contrôle et le débogage du logiciel se fait via GDB.

\subsubsection{Tests des données envoyées}

Pour vérifier que notre logiciel envoie des données correctes sur le bus à notre
IP, via le DMA, il faut regarder les données en entrée de notre IP.
Pour cela, nous avons légèrement modifié le code VHDL afin de bloquer les
données dans la première FIFO tant qu'il n'y a pas de lecture dans
un certain registre. Ainsi la lecture dans ce registre provoque la sortie de
la première donnée de la FIFO. En sortant les données une à une, nous pouvons
vérifier que le logiciel envoie les bonnes données.

\subsubsection{Tests des données en sortie de premier niveau de neurones}

Pour vérifier que le premier niveau de neurones est correctement configuré et
renvoie les données attendues en sortie, nous avons recours au même procédé que
dans la partie précédente, mais en sortie de la FIFO entre le premier niveau
de neurones et l'étage de recodage.
Ainsi, nous pouvons observer les données en sortie du premier niveau de neurones.
En envoyant des images complètement vides(pixels à 0) sauf pour un pixel à 1,
nous pouvons récupérer l'ensemble de la configuration du premier niveau et
vérifier qu'elle correspond à ce que nous attendons.

\subsubsection{Tests des données en sortie de l'étage de recodage}

Pour vérifier que l'étage de recodage est correctement configuré et renvoie les
données attendues en sortie, nous avons recours au même procédé que dans les
tests sur carte précédents. Nous récupérons les données en sortie de l'étage
de recodage via la FIFO qui est connectée à sa sortie.
De plus, en modifiant le VHDL, il est possible de charger des données dans la
FIFO d'entrée de l'étage de recodage de façon manuelle, en écrivant des données
dans un certain registre. Grâce à cela, nous pouvons vérifier que l'étage
est correctement configuré et fonctionne comme attendu.

\subsubsection{Tests des données en sortie de deuxième niveau de neurones}

Pour vérifier que le dernier étage de neurones est correctement configuré et
renvoie les données attendues en sortie, on procède de la même manière que pour
l'étage de recodage.
On vérifie les données en sortie du niveau via un registre, et au besoin on
contrôle son entrée via un autre registre. Grâce à cela, nous pouvons vérifier
que l'étage est correctement configuré et fonctionne comme attendu.


\subsubsection{Conclusion}

Grâce aux tests réalisés sur carte, nous pouvons terminé de valider notre IP.
En effet, après avoir vérifié le comportement de chaque composant en simulation,
nous pouvons voir comment cela se passe sur carte avec le logiciel.
Certains problèmes logiciels comme matériels sont apparus lors de cette phase,
et ont été assez difficiles à analyser correctement à cause du manque
d'information par rapport à une simulation classique sur ordinateur.
