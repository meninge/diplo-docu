De nos jours, les réseaux de neurones artificiels prennent de plus en plus d'importance
dans la vie de tous les jours. Traitement d'images, reconnaissance vocale, traitements lexicaux sont
des applications qui pourraient être intégrées dans des systèmes embarqués. Pour cela, il est possible 
de les implémenter sur CPU ou GPU, mais la consommation et la vitesse de traitement deviendraient vite problématiques.

Créer un composant électronique implémentant un réseau de neurones réduirait drastiquement sa consommation
et améliorerait possiblement la vitesse du réseau par rapport à un CPU. De plus, rendre paramétrable dynamiquement le
composant lui permettrait de s'adapter à de multiples applications.\\

Le projet "Réseau de neurones sur FPGA" s'incrit dans ce cadre. Sous le tutorat
de Frédéric Pétrot et Adrien Prost-Boucle, nous devons créer un tel composant,
et tester ses performances en vue de le comparer à des systèmes existants tels
que la puce Spinnaker, ou TrueNorth d'IBM, ou bien de systèmes en développement
tel que les réseaux de neurones ternaires du laboratoire TIMA. Une fois
implémenté et validé, nous utiliserons une carte FPGA Zedboard pour tester notre
composant sur une application classique de reconnaissance de chiffres manuscrits,
en utilisant la base de données MNIST.
