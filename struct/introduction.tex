De nos jours, les réseaux de neurones artificiels prennent de plus en plus d'importance
dans la vie de tous les jours. Traitement d'images, reconnaissance vocale, traitements lexicaux sont
des applications qui pourraient être intégrées dans des systèmes embarqués. Pour cela, il est possible 
de les implémenter sur CPU ou GPU, mais la consommation et la vitesse de traitement deviendraient vite problématiques.

Créer un composant électronique implémentant un réseau de neurones réduirait drastiquement sa consommation
et améliorerait possiblement la vitesse du réseau par rapport à un CPU. De plus, rendre paramétrable dynamiquement le
composant lui permettrait de s'adapter à de multiples applications.

A COMPLETER
